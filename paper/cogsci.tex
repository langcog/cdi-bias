% Template for Cogsci submission with R Markdown

% Stuff changed from original Markdown PLOS Template
\documentclass[10pt, letterpaper]{article}

\usepackage{cogsci}
\usepackage{pslatex}
\usepackage{float}
\usepackage{caption}

% amsmath package, useful for mathematical formulas
\usepackage{amsmath}

% amssymb package, useful for mathematical symbols
\usepackage{amssymb}

% hyperref package, useful for hyperlinks
\usepackage{hyperref}

% graphicx package, useful for including eps and pdf graphics
% include graphics with the command \includegraphics
\usepackage{graphicx}

% Sweave(-like)
\usepackage{fancyvrb}
\DefineVerbatimEnvironment{Sinput}{Verbatim}{fontshape=sl}
\DefineVerbatimEnvironment{Soutput}{Verbatim}{}
\DefineVerbatimEnvironment{Scode}{Verbatim}{fontshape=sl}
\newenvironment{Schunk}{}{}
\DefineVerbatimEnvironment{Code}{Verbatim}{}
\DefineVerbatimEnvironment{CodeInput}{Verbatim}{fontshape=sl}
\DefineVerbatimEnvironment{CodeOutput}{Verbatim}{}
\newenvironment{CodeChunk}{}{}

% cite package, to clean up citations in the main text. Do not remove.
\usepackage{apacite}

% KM added 1/4/18 to allow control of blind submission


\usepackage{color}

% Use doublespacing - comment out for single spacing
%\usepackage{setspace}
%\doublespacing


% % Text layout
% \topmargin 0.0cm
% \oddsidemargin 0.5cm
% \evensidemargin 0.5cm
% \textwidth 16cm
% \textheight 21cm

\title{Words aren't created equal: Investigating bias on the CDI}


\author{{\large \bf Morton Ann Gernsbacher (MAG@Macc.Wisc.Edu)} \\ Department of Psychology, 1202 W. Johnson Street \\ Madison, WI 53706 USA \AND {\large \bf Sharon J.~Derry (SDJ@Macc.Wisc.Edu)} \\ Department of Educational Psychology, 1025 W. Johnson Street \\ Madison, WI 53706 USA}


\begin{document}

\maketitle

\begin{abstract}
Children's early language skill has been linked to later educational
outcomes, making it important to accurately measure early language.
Parent-reported instruments such as the Communicative Development
Inventories (CDIs) have been shown to provide valid, consistent measures
of children's aggregate early language skill. However, CDIs are
predominantly comprised of hundreds of vocabulary items, some of which
may not be heard (and thus learned) equally often by children of varying
backgrounds. Here, we use a database of American English CDIs to
identify words that show strong bias for particular groups of children,
along the axes of sex (male vs.~female), race/ethnicity (white
vs.~non-white), and socioeconomic status (high vs.~low). For each axis,
we identify dozens of strongly biased items, and show that eliminating
these items reduces the expected ability difference between groups. For
sex, we consider how to propose replacement words that may show less
bias, on the basis of their relatively equal frequency in adult speech
directed to male and female children.

\textbf{Keywords:}
language development; early vocabulary; measuring instrument bias;
\end{abstract}

\hypertarget{introduction}{%
\section{Introduction}\label{introduction}}

introduce importance of measuring early language development (e.g.,
links to later educational outcomes)

introduce CDI

predictive validity: Fenson et al.~1994, Bornstein et al.~2012, Duff et
al.~2015

discuss potential for bias

what does between-group bias look like?

IRT - what is a good measure?

\hypertarget{fundamental-dif-problem}{%
\subsection{Fundamental DIF Problem}\label{fundamental-dif-problem}}

what is our goal for measuring vocabulary? - identifying language
delays, predicting later reading, or talking\ldots{}

how you select items influences how well you achieve these goals (and
what bias you find)

We take the approach put forward by Stenhaug, Frank, \& Domingue (2021):
GLIMMER plots based on the 1-parameter logistic are sufficient to
identify bias, without potentially obscuring DIF in a more complex
model's additional parameters.

Differential item functioning (DIF) is a technique in IRT used to
identify items that show bias against demographic groups. DIF is a
statistical characteristic of an item that shows the extent to which the
item might operate differently or measure varying abilities for
subgroups and members of separate demographic groups. DIF is, however, a
challenging metric for identifying bias for a couple reasons. Firstly,
DIF is extremely hard to accurately root out. The process of finding DIF
involves using the test you used to do IRT analysis in order to look for
bias within the very same test. Secondly, the presence of DIF does not
necessarily indicate the presence of bias; it can indicate that one
demographic has a higher average ability level than the other. These two
issues are the basis upon which we get the Fundamental DIF
Identification problem.

This problem can be be understood through a concrete example that is
well-known from research on early language learning: consider the fact
that females show a larger vocabulary than males across early
development {[}see Frank, Braginsky, Yurovsky, \& Marchman (2021) Ch.
6?; OTHER REFS{]}. The question however is whether or not girls actually
have a higher ability level or if the tests have words that are bias
toward girls ie. are easier for girls, or perhaps the test is biased
toward boys but the natural ability of girls is enough to overcome that
bias. To answer this question we need to know the ability levels of
girls and boys in order to confirm that if a word is learned earlier by
girls it is simply because of an ability difference rather than a bias
inherent to the word. The problem arises from the fact that we measure
ability level using the very same test that we are trying to check for
biased words. If the boys and girls were of the same ability level this
wouldn't be an issue but given the evidence showing that girls have a
higher ability level, we need to know the difference in ability so that
when we find DIF for a specific word can be sure it is outside of the
expecting DIF that results naturally from their difference in ability.

\hypertarget{methods}{%
\section{Methods}\label{methods}}

\hypertarget{dataset}{%
\subsection{Dataset}\label{dataset}}

We analyze American English CDI: Words \& Sentences administrations from
5520 children aged 16 to 30 from Wordbank (Frank, Braginsky, Yurovsky,
\& Marchman, 2017; Frank et al., 2021).

\hypertarget{participants}{%
\subsubsection{Participants}\label{participants}}

Research was conducted over 5520 participants, whose data was parent
reported and collected by the Wordbank Project. Unlike other projects
Wordbank relies on the kindness of others to contribute their data which
means that often meta-data for some sets is missing. Even so, the
project now has data over 29 different languages on children from ages
8-30 months old, though our study focused entirely on American English
and Children 16-30 months. When they received complete information, the
Wordbank data set collected demographic data consisting of Birth Order,
Race/Ethnicity, Sex and Mothers Education. We focused on comparing data
between demographic groups on the axes of Sex, Ethnicity and Mother's
Education, a proxy for socioeconomic status (henceforth, SES). Our data
included 1989 female and 2105 male participants, and an additional 1426
participants whose sex remained unreported. In terms of Ethnicity data
was recorded from 2202 white, 67 Asian, 222 Black, 131 Hispanic, and 93
``Other'' participants as well as 2805 unreported. Given our
distribution of ethnicities, for our analysis we split participants into
White (2202) and Non-White (513). Finally, we split participants into
high and low SES, of which we had 4973 high SES participants,
corresponding to mothers' education of some college or higher and 547
Low SES participants.

\hypertarget{one-column-images}{%
\subsection{One-column images}\label{one-column-images}}

\hypertarget{acknowledgements}{%
\section{Acknowledgements}\label{acknowledgements}}

Place acknowledgments (including funding information) in a section at
the end of the paper.

\hypertarget{references}{%
\section{References}\label{references}}

\setlength{\parindent}{-0.1in} 
\setlength{\leftskip}{0.125in}

\noindent

\hypertarget{refs}{}
\leavevmode\hypertarget{ref-Frank2017}{}%
Frank, M. C., Braginsky, M., Yurovsky, D., \& Marchman, V. A. (2017).
Wordbank: An open repository for developmental vocabulary data.
\emph{Journal of Child Language}, \emph{44}(3), 677--694.

\leavevmode\hypertarget{ref-Frank2021}{}%
Frank, M. C., Braginsky, M., Yurovsky, D., \& Marchman, V. A. (2021).
\emph{Variability and consistency in early language learning: The
wordbank project}. MIT Press.

\leavevmode\hypertarget{ref-stenhaug2021treading}{}%
Stenhaug, B., Frank, M. C., \& Domingue, B. (2021). Treading carefully:
Agnostic identification as the first step of detecting differential item
functioning.

\bibliographystyle{apacite}


\end{document}
