% Template for Cogsci submission with R Markdown

% Stuff changed from original Markdown PLOS Template
\documentclass[10pt, letterpaper]{article}

\usepackage{cogsci}
\usepackage{pslatex}
\usepackage{float}
\usepackage{caption}

% amsmath package, useful for mathematical formulas
\usepackage{amsmath}

% amssymb package, useful for mathematical symbols
\usepackage{amssymb}

% hyperref package, useful for hyperlinks
\usepackage{hyperref}

% graphicx package, useful for including eps and pdf graphics
% include graphics with the command \includegraphics
\usepackage{graphicx}

% Sweave(-like)
\usepackage{fancyvrb}
\DefineVerbatimEnvironment{Sinput}{Verbatim}{fontshape=sl}
\DefineVerbatimEnvironment{Soutput}{Verbatim}{}
\DefineVerbatimEnvironment{Scode}{Verbatim}{fontshape=sl}
\newenvironment{Schunk}{}{}
\DefineVerbatimEnvironment{Code}{Verbatim}{}
\DefineVerbatimEnvironment{CodeInput}{Verbatim}{fontshape=sl}
\DefineVerbatimEnvironment{CodeOutput}{Verbatim}{}
\newenvironment{CodeChunk}{}{}

% cite package, to clean up citations in the main text. Do not remove.
\usepackage{apacite}

% KM added 1/4/18 to allow control of blind submission


\usepackage{color}

% Use doublespacing - comment out for single spacing
%\usepackage{setspace}
%\doublespacing


% % Text layout
% \topmargin 0.0cm
% \oddsidemargin 0.5cm
% \evensidemargin 0.5cm
% \textwidth 16cm
% \textheight 21cm

\title{Investigating Item Bias}


\author{{\large \bf Morton Ann Gernsbacher (MAG@Macc.Wisc.Edu)} \\ Department of Psychology, 1202 W. Johnson Street \\ Madison, WI 53706 USA \AND {\large \bf Sharon J.~Derry (SDJ@Macc.Wisc.Edu)} \\ Department of Educational Psychology, 1025 W. Johnson Street \\ Madison, WI 53706 USA}


\begin{document}

\maketitle

\begin{abstract}
Include no author information in the initial submission, to facilitate
blind review. The abstract should be one paragraph, indented 1/8 inch on
both sides, in 9\textasciitilde point font with single spacing. The
heading `Abstract' should be 10\textasciitilde point, bold, centered,
with one line of space below it. This one-paragraph abstract section is
required only for standard six page proceedings papers. Following the
abstract should be a blank line, followed by the header `Keywords' and a
list of descriptive keywords separated by semicolons, all in
9\textasciitilde point font, as shown below.

\textbf{Keywords:}
language development;
\end{abstract}

\hypertarget{introduction}{%
\section{Introduction}\label{introduction}}

\begin{itemize}
\item
  introduce importance of measuring early language development (e.g.,
  links to later educational outcomes)
\item
  introduce CDI
\item
  predictive validity: Fenson et al.~1994, Bornstein et al.~2012, Duff
  et al.~2015
\item
  discuss potential for bias
\item
  what does between-group bias look like?
\item
  IRT - what is a good measure?
\end{itemize}

\hypertarget{fundamental-dif-problem}{%
\subsection{Fundamental DIF Problem}\label{fundamental-dif-problem}}

\begin{itemize}
\tightlist
\item
  what is our goal for measuring vocabulary? - identifying language
  delays, predicting later reading, or talking\ldots{}
\item
  how you select items influences how well you achieve these goals (and
  what bias you find) We take the approach put forward by
  (\textbf{Stenhaug2021?}): GLIMMER plots based on the 1-parameter
  logistic are sufficient to identify bias, without potentially
  obscuring DIF in a more complex model's additional parameters.
\end{itemize}

\hypertarget{methods}{%
\section{Methods}\label{methods}}

\hypertarget{dataset}{%
\subsection{Dataset}\label{dataset}}

We analyze American English CDI: Words \& Sentences administrations from
5520 children aged 16 to 30 from Wordbank (\textbf{Frank2016?};
\textbf{Frank2021?}).

\hypertarget{participants}{%
\subsubsection{Participants}\label{participants}}

Research was conducted over 5520 participants, whose data was parent
reported and collected by the Wordbank Project. Unlike other projects
Wordback relies on the kindness of others to contribute their data which
means that often meta-data for some sets is missing. Even so, the
project now has data over 29 different languages on children from ages
8-30 months old, though our study focused entirely on American English
and Children 16-30 months. When they received complete information, the
Wordbank data set collected demographic data consisting of Birth Order,
Race/Ethnicity, Sex and Mothers Education. For our study we focused on
the comparing data between demographic groups on the axis of Sex,
Ethnicity and Mothers Education, which was a proxy for social economic
status. Out data included 1989 female and 2105 male participants, and an
additional 1426 participants whose sex remained unreported. In terms of
Ethnicity data was recorded from 2202 white, 67 Asian, 222 Black, 131
Hispanic, and 93 ``Other'' participants as well as 2805 unreported.
Given our distribution of ethnicities, for our analysis we split
participants into White (2202) and Non-White (513). Finally, we split
participants into high and low SES, of which we had NA high SES
participants, corresponding to mothers' education of some college or
higher and NA Low SES participants.

\bibliographystyle{apacite}


\end{document}
